\documentclass{fennix_plantilla/fennix}
\usepackage{listings}
\usepackage{color}
\usepackage{hyperref}

\definecolor{mygreen}{rgb}{0,0.6,0}
\definecolor{mygray}{rgb}{0.5,0.5,0.5}
\definecolor{mymauve}{rgb}{0.58,0,0.82}

\lstset{
	language=C,
	numbers=left,
	backgroundcolor=\color{white},
	basicstyle=\footnotesize,
	breakatwhitespace=false,         % sets if automatic breaks should only happen at whitespace        
	breaklines=true,                 % sets automatic line breaking
	captionpos=b,                    % sets the caption-position to bottom
	commentstyle=\color{mygreen},    % comment style
	frame=single,                    % adds a frame around the code
	keywordstyle=\color{blue},       % keyword style
	otherkeywords={*,...},            % if you want to add more keywords to the set
	numbers=left,                    % where to put the line-numbers; possible values are (none, left, right)
	numbersep=5pt,                   % how far the line-numbers are from the code
	numberstyle=\tiny\color{mygray}, % the style that is used for the line-numbers
	rulecolor=\color{black},
	showstringspaces=false,          % underline spaces within strings only
	stepnumber=2,                  
	stringstyle=\color{mymauve},     % string literal style
	tabsize=2,                       % sets default tabsize to 2 spaces
	caption=\lstname                   % show the filename of files included with \lstinputlisting; also try caption instead of title
}

\begin{document}
\title{Introducción a GDB}
\author{stakewinner00}
\date{\today -- alpha}
\maketitle
\begin{resumen}
En este texto se estudiara como depurar un programa usando GDB. Se explicara tanto desde el punto de vista de un programador, como el de un hacker. También se explicarán métodos usado para proteger un binario de ser analizado con GDB.
\end{resumen}

\begin{requisitos}
\begin{itemize}
\item Entender lenguaje ensamblador x86
\item Entender códigos en C/C++
\item Tener instalado GDB
\item Tener instalado GCC
\end{itemize}
\end{requisitos}

\section{Introducción}
Los programas a veces no funcionan como esperamos, a veces dan fallos de segmentación, a veces entran en bucles infinitos, a veces acaban consumiendo demasiada memoria o CPU. Cuando nos encontramos en situaciones como estas tenemos varias opciones, o nos quedamos de brazos cruzado, o buscamos por internet para hacer copy paste y esperar que funcione, o si tenemos tiempo, nos dedicamos a ver como funciona el programa por dentro y mirar porque se comporta de esa forma. 

Para facilitarnos la vida existen distintas herramientas que nos aportan información sobre la ejecución de un programa, algunas de ellas sirven para bajar a nivel ensamblador y mirar que esta ejecutando el procesador. Estas herramientas se denominan depuradores ya que se usan para testear y depurar programas.

\subsection{¿Qué es un depurador?}
Un depurador (o debugger en ingles) es un programa que como dijimos nos permite analizar el comportamiento de un programa. Algunos depuradores, como GDB nos permiten mostrarnos la línea  de código original a alto nivel que se esta ejecutando además de mostrarnos el código ensamblador y el estado de los registros.

\subsection{¿Quién usa los depuradores?}
Un depurador puede ser usado por cualquier interesado en entender el funcionamiento de un ejecutable. Por ejemplo un programador para saber porque su programa no funciona, un investigador de seguridad que esta buscando vulnerabilidades para crear un exploit como prueba de concepto, o incluso un crackear para analizar como un programa valida un serial y evadir el checkeo. \\ 

Normalmente en el caso del programador es distinto de los otros dos casos ya que dispone del código fuente y puede generar "símbolos de depuración". Esta información ayudara al depurador a reconocer variables, funciones, y en definitiva ''entender'' mejor el ejecutable.\\ En los otros dos casos normalmente no se dispone del código fuente y además la poca información que podría contener normalmente fue cuidadosamente eliminada, se explicara más adelante este tema.

\subsection{¿Por qué usar depuradores?}
Bien se podrían depurar los programas a base de llamadas a printf() o similar, pero hacer esto tiene una serie de inconvenientes. En primer lugar se tiene menos información que un depurador con el que puedes ver registros y demás, en segundo lugar es más lento ya que tendrás que recompilar varias veces, en tercer lugar el código añadido para depurar puede variar el funcionamiento original del programa.

\subsection{Interfaces gráficas de GDB}
Existen varias interfaces gráficas para GDB (como DDD), sin embargo en este texto se usara principalmente el modo consola. Las interfaces gráficas de GDB suelen dejar bastante que desear y al final es mucho más didáctico aprender a usarlo por consola. Además habrá ocasiones en los que no sea posible usar las interfaces gráficas.
\section{Primeros pasos}
\emph{Nota: Los códigos fuente que se usaran están en la carpeta ''ejemplos/<capítulo>''. }\\
En este capítulo haremos un repaso sobre que son y como colocar, breakpoints, como ver los registros, leer el stack, imprimir variables, y como configurar GDB. En resumen, un hello world en GDB


\subsection{Hello world GDB}
Una vez ya tengamos instalado GCC y GDB procederemos a compilar y analizar un simple Hello World. \lstinputlisting[language=C]{ejemplos/2/hello.c}
Como se ve el programa solo imprime "Hello world GDB" por pantalla. \\
Compilemos el código usando gcc, \textbf{gcc hello.c -ggdb -o hello}. La opción o flag \emph{-ggdb} es importante ya que generara los \emph{debugging symbols}\\
Una vez compilado debemos abrir este ejecutable con GDB. Para hacerlo podemos especificar el nombre como un parámetro, o usar el comando \emph{file} seguido del nombre del ejecutable. 

Si ejecutamos el comando \emph{gdb} podremos ver un mensaje introductorio y información sobre la licencia. 
\begin{verbatim}
GNU gdb (GDB) 7.8.50.20150104-cvs
Copyright (C) 2015 Free Software Foundation, Inc.
License GPLv3+: GNU GPL version 3 or later <http://gnu.org/licenses/gpl.html>
This is free software: you are free to change and redistribute it.
There is NO WARRANTY, to the extent permitted by law.  Type "show copying"
and "show warranty" for details.
This GDB was configured as "x86_64-unknown-linux-gnu".
Type "show configuration" for configuration details.
For bug reporting instructions, please see:
<http://www.gnu.org/software/gdb/bugs/>.
Find the GDB manual and other documentation resources online at:
<http://www.gnu.org/software/gdb/documentation/>.
For help, type "help".
Type "apropos word" to search for commands related to "word".
(gdb) 
\end{verbatim}

Una vez ya iniciado GDB podemos usar el comando \emph{file} como se dijo antes. Como vemos a continuación, al ejecutar este comando GDB nos dice \textbf{Reading symbols from hello...done.}, esto es porque antes al compilarlo usamos el flag \emph{-ggdb} que genero la información de depuración (símbolos) que ahora GDB leyó. 
\begin{verbatim} 
(gdb) file hello
Reading symbols from hello...done.
(gdb)
\end{verbatim}
Ahora gracias a la información de depuración y que tenemos el código fuente, podemos ejecutar el comando \emph{list} o \emph{l} (l es una abreviación para list). Este comando nos mostrara el código fuente de 10 en 10 líneas. Para listar rangos concretos se le puede pasar como parámetro la primera y última línea a listar. Por ejemplo para listar entre la 4 y la 6 escribiríamos \emph{l 4,6}. Como vemos a continuación muestra las tres líneas.
\begin{verbatim}
(gdb) l 4,6
4 {
5   puts("Hello world GDB");
6 }
(gdb)
\end{verbatim}
Una vez hayamos terminado nuestra primera experiencia con GDB, podemos salir con el comando \emph{quit} (o su versión abreviada \emph{q}) o bien con la combinación Ctrl-D.

\subsection{Jugando con variables}
A veces puede interesarnos leer una variable, o cambiarle el valor. Por ejemplo para ver el valor apuntado por un puntero. \\Para poder estudiar el estado de las variables en un cierto punto usaremos breakpoints (puntos donde el programa parara). En esta sección solo veremos un ejemplo, más adelante estudiaremos los distintos tipos de breakpoints y algunos de sus usos. \\

\emph{El código que se usara esta en la carpeta \textbf{ejemplos/2/vars.c}}\\

En el código podemos ver que hay dos variables, una es un puntero.\\ \lstinline|unsigned int x,*y;|\\
También podemos ver que hay 2 llamadas a scanf, la segunda tiene un fallo claro que producirá un fallo de segmentación al intentar escribir en el valor apuntado por y (ya que y no contendrá una dirección válida).\\
\lstinline|scanf("%d", y);|\\
Como antes, compilamos con el flag \emph{-ggdb} y lo abrimos con GDB. Ahora lo abriremos en modo ''silencioso'' (sin imprimir información de GDB como la licencia y demás) invocando a GDB con el flag \emph{-q}, \emph{--quiet} o \emph{--silent}. También podemos pasarle el nombre del ejecutable, así lo cargara directamente en GDB sin tener que ejecutar el comando \emph{file}.
\begin{verbatim}
$ gdb -q vars
Reading symbols from vars...done.
(gdb)
\end{verbatim}
Una vez abierto podemos listar el código con el comando \emph{list} como se dijo anteriormente. Vemos que si escribimos \textbf{l} lista las 10 primeras líneas como dijimos, y si volvemos a apretar enter, sin escribir ningún comando, mostrara las 6 líneas restantes. Como se puede deducir, GDB recuerda el último comando y si presionamos enter directamente lo repite.\\

Para poder analizar el valor de la variable \textbf{y} debemos poner un breakpoint antes de ejecutarse la línea 12. Para poner un breakpoint en ese segundo scanf simplemente podemos escribir el comando \emph{break 12} o \emph{b 12}. 
\begin{verbatim}
(gdb) b 12
Breakpoint 1 at 0x4005af: file vars.c, line 12.
(gdb) 
\end{verbatim}
Como vemos nos dice en que posición de memoria coloco el breakpoint, en que archivo y en que línea. También nos dice que es el breakpoint número 1. \\
Ahora que ya tenemos el breakpoint podemos ejecutar el programa con el comando \emph{run} (o ''r''). Primero se ejecutara el primer scanf, escribimos algún número para que se guarde en la variable \textbf{x} y luego saltara el breakpoint.
\begin{verbatim}
(gdb) r
Starting program: /tmp/codes/1/vars 
5 

Breakpoint 1, main (argc=1, argv=0x7fffffffeb08) at vars.c:12
12	  scanf("%d", y);
(gdb) 
\end{verbatim}
Como vemos, GDB nos informa de la función en la que nos encontramos y el valor de los parámetros. En este caso argc vale 1 y argv 0x7fffffffeb08.
Ahora por ejemplo podemos jugar con el comando \emph{print} (o la abreviación''p'') para mirar el valor de las variables. Por ejemplo si miramos la variable \textbf{x} nos mostrara 5, ya que le acabamos de asignar un valor. 
\begin{verbatim}
(gdb) p x
$1 = 5
(gdb) p y
$2 = (unsigned int *) 0x0
(gdb)
\end{verbatim}
Como vemos \textbf{x} vale 5 y \textbf{y} vale 0. En este caso en el siguiente scanf se producirá un error al intentar acceder a la posición 0. \\ Para que no diera fallo de segmentación podríamos jugar a modificar el valor de \textbf{y} y asignarle la dirección de \textbf{x}. De esta forma sobre-escribiremos el valor de \textbf{x}. \\ Para modificar un valor a una variable podemos usar el comando \emph{set var}.
\begin{verbatim}
(gdb) p &x
$3 = (unsigned int *) 0x7fffffffea14
(gdb) set var y= 0x7fffffffea14
(gdb) p y
$4 = (unsigned int *) 0x7fffffffea14
(gdb) p *y
$5 = 5
(gdb)
\end{verbatim}
Como vemos simplemente hicimos que \textbf{y} apuntara al valor de \textbf{x}.
Ahora ya podríamos usar el comando \emph{continue} (o su abreviación ''c'') para seguir con le ejecución.
\begin{verbatim}
(gdb) c
Continuing.
2   
Suma: 4
[Inferior 1 (process 28169) exited normally]
(gdb) 
\end{verbatim}
Como vemos le dimos el valor 2 al segundo scanf sobre-escribiendo \textbf{x}, por tanto la suma es 4. GDB ya nos dice que el programa finalizo normalmente con el PID (ID del proceso) 28169.\\

Otra forma más ''elegante'' de corregir temporalmente este error podría ser reservar memoria para \textbf{y}. GDB permite hacer llamadas a funciones como malloc. Así que simplemente podríamos llamar a malloc(4) y modificar la variable \textbf{y} para que apunte a esa dirección.
\begin{verbatim}
(gdb) call malloc(4)
$1 = 6295568
(gdb) p y
$2 = (unsigned int *) 0x0
(gdb) set var y = 6295568
(gdb) p *y
$3 = 0
(gdb) c
Continuing.
2
Suma: 7
[Inferior 1 (process 29501) exited normally]
(gdb) 
\end{verbatim}
Como vemos al principio \textbf{y} no apunta a nada y luego modificamos su valor por la dirección devuelta por malloc. Como vemos la memoria que reservo malloc contiene un 0. Continuamos con la ejecución y escribimos un 2 en \textbf{*y}. Ahora sí que suma el primer 5 introducido en \textbf{x} más el 2 de \textbf{*y}.


\subsection{Modificando los registros}
En esta sección veremos un ejemplo de como podemos leer y modificar registros para que ejecute una función que en teoría no debería ejecutarse.\\

\emph{El código que se usara esta en \textbf{ejemplos/2/registros.c}}\\

Una vez compilemos y carguemos el ejecutable en GDB ponemos un breakpoint en la función main y le damos run.
\begin{verbatim}
(gdb) break main
Breakpoint 1 at 0x400520: file registros.c, line 9.
(gdb) r
Starting program: /tmp/codes/1/registros 

Breakpoint 1, main () at registros.c:9
9	{}
(gdb) 
\end{verbatim}

Para desensamblar cerca de la posición actual escribiremos el comando \emph{disassemble} o su abreviación \emph{disas}.
\begin{verbatim}
(gdb) disas
Dump of assembler code for function main:
0x0000000000400517 <+0>:	push   %rbp
0x0000000000400518 <+1>:	mov    %rsp,%rbp
0x000000000040051b <+4>:	mov    $0x0,%eax
=> 0x0000000000400520 <+9>:	pop    %rbp
0x0000000000400521 <+10>:	retq   
End of assembler dump.
(gdb) 
\end{verbatim}
Como vemos nos indica con \textbf{=>} la dirección a la que apunta el registro EIP (o RIP en x86-64). Como vemos el desensamblado esta con la sintaxis at\&t y no de intel, cuando toquemos los archivos de configuración de GDB veremos como podemos cambiarlo de forma permanente.

Bien, ahora si queremos que se ejecute la shell debemos cambiar el registro rip para que apunte a la dirección correspondiente. Para eso primero tenemos que saber la dirección de la función shell. Para eso podemos usar los comandos vistos anteriormente.
\begin{verbatim}
(gdb) p shell
$1 = {void (void)} 0x400506 <shell>
(gdb)
\end{verbatim}
Ahora que sabemos donde esta solo tenemos que cambiar el registro rip. Primero miremos como tenemos los registros, para hacerlo existe el comando \texttt{info}. En concreto para ver los registros podemos usar \texttt{info registers } (o \texttt{i r})
\begin{verbatim}
(gdb) i r
rax            0x0	0
rbx            0x0	0
rcx            0x0	0
rdx            0x7fffffffeb08	140737488349960
rsi            0x7fffffffeaf8	140737488349944
rdi            0x1	1
rbp            0x7fffffffea10	0x7fffffffea10
rsp            0x7fffffffea10	0x7fffffffea10
r8             0x7ffff7dd6dd0	140737351871952
r9             0x7ffff7dea6d0	140737351952080
r10            0x833	2099
r11            0x7ffff7a57760	140737348204384
r12            0x400410	4195344
r13            0x7fffffffeaf0	140737488349936
r14            0x0	0
r15            0x0	0
rip            0x400520	0x400520 <main+9>
eflags         0x246	[ PF ZF IF ]
cs             0x33	51
ss             0x2b	43
ds             0x0	0
es             0x0	0
fs             0x0	0
gs             0x0	0
(gdb)
\end{verbatim}
Cambiamos el registro rip por la dirección correspondiente.
\begin{verbatim}
(gdb) set var $rip = 0x400506
(gdb) i r rip
rip            0x400506	0x400506 <shell>
(gdb)
\end{verbatim}
Es importante destacar que los registros van con el símbolo del dolar ''\$'' delante. También se puede apreciar que al comando \texttt{i r} se le puede especificar el registro a imprimir, en este caso rip. Como el comando \texttt{i r} ya se sabe que trata registros no es necesario ponerle el símbolo del dolar. Ahora si continuamos la ejecución con el comando \texttt{continue} tendremos nuestra shell.
\begin{verbatim}
(gdb) c
Continuing.
$ exit
[Inferior 1 (process 22319) exited normally]
(gdb)
\end{verbatim}

Otra forma más rápida de hacer lo mismo hubiera sido usar el comando \texttt{set var} después de poner el breakpoint y ejecutar.
\begin{verbatim}
(gdb) set var $rip = shell
(gdb) c
Continuing.
$
\end{verbatim}


\subsection{Crackeando binarios}
Una de las aplicaciones que puede tener el uso de un depurador puede ser la de crackear un binario evadiendo la licencia o el serial. En esta sección veremos un ejemplo de laboratorio. \\
 
\emph{El código del binario esta en \textbf{ejemplos/2/crackme.c}}\\

Esta vez compilamos el binario sin los flags de depuración (para hacerlo más realista).\\
Lo cargamos en GDB como de costumbre y ponemos un breakpoint en la función main, le damos run y desensamblamos. Como vimos por defecto se usa la sintaxis at\&t, para cambiarlo temporalmente podemos escribir \textbf{set disassembly-flavor intel}. \\
En el código desensamblado se pueden observar 3 llamadas a la función puts. La primera simplemente imprime la string ''Introduzca el serial:'' , las otras dos que se muestran a continuación son más interesantes.
\begin{verbatim}
 0x00000000004005b0 <+74>:	jne    0x4005be <main+88>
 0x00000000004005b2 <+76>:	mov    edi,0x40066e
 0x00000000004005b7 <+81>:	call   0x400430 <puts@plt>
 0x00000000004005bc <+86>:	jmp    0x4005c8 <main+98>
 0x00000000004005be <+88>:	mov    edi,0x400677
 0x00000000004005c3 <+93>:	call   0x400430 <puts@plt>
 0x00000000004005c8 <+98>:	mov    eax,0x0
\end{verbatim}
Como vemos depende de una cierta condición imprimirá la string de la dirección \textbf{0x40066e} y sino de la \textbf{0x400677}. Si imprimimos el valor de las dos direcciones podemos ver que la primera dice ''Correcto'' y la segunda ''Incorrecto''.
\begin{verbatim}
(gdb) p (char*)0x40066e
$2 = 0x40066e "Correcto"
(gdb) p (char*)0x400677
$3 = 0x400677 "Incorrecto"
(gdb)
\end{verbatim}
\emph{Obsérvese que se puede especificar el tipo, por ejemplo aquí queremos la string, no los valores ASCII}\\

Por tanto una primera solución podría ser nopear el salto condicional para que nunca saltara, así siempre imprimiría el mensaje ''Correcto''. Otra solución más bonita es mirar con que compara y generar el serial correcto.
Justo después de la llamada a scanf hay una secuencia de comparaciones. Para mirar exactamente que valores mueve pondremos un breakpoint en el scanf.
\begin{verbatim}
 0x0000000000400589 <+35>:	call   0x400460 <__isoc99_scanf@plt>
\end{verbatim}
En este caso en la dirección 0x0000000000400589 o lo que es lo mismo, en *main + 35.
\begin{verbatim}
 (gdb) b *main+35
 Breakpoint 2 at 0x400589
 (gdb) 
\end{verbatim}
Una vez se ejecute la instrucción
\begin{verbatim}
0x000000000040058e <+40>:	movzx  eax,BYTE PTR [rbp-0x10]
\end{verbatim}
podemos mirar que valor tiene el registro EAX. En este caso el serial introducido fue ''123456''. Como vemos si imprimimos el valor de EAX equivale a la primera letra del serial, en este caso un ''1'' como se puede ver.
\begin{verbatim}
(gdb) p (char)$rax
$20 = 49 '1'
(gdb)
\end{verbatim}
Luego vemos que compara esa primera letra y si no es igual a 0x35 (''5'') salta al mensaje de ''Incorrecto''.
\begin{verbatim}
0x0000000000400592 <+44>:	cmp    al,0x35
=> 0x0000000000400594 <+46>:	jne    0x4005be <main+88>
\end{verbatim}
\emph{0x4005be es la dirección donde pone el mensaje de ''Incorrecto''.} 
Ya tenemos el primer carácter del serial, ''5''. \\
Haciendo lo mismo vemos que la siguiente comparación es contra 0x61 (letra ''a'') y la siguiente 0x37 (''7''), así que los tres primeros caracteres son ''5a7''. 
Finalmente podemos ver que compara el registro dl y el registro al. 
\begin{verbatim}
0x00000000004005a6 <+64>:	movzx  edx,BYTE PTR [rbp-0xd]
0x00000000004005aa <+68>:	movzx  eax,BYTE PTR [rbp-0xc]
0x00000000004005ae <+72>:	cmp    dl,al
0x00000000004005b0 <+74>:	jne    0x4005be <main+88>
\end{verbatim} 
El registro \textbf{dl} contendrá la penúltima letra a comparar y \textbf{al} la última, por tanto los dos últimos caracteres del serial tienen que ser los mismos. Así que ''5a700' o ''5a7zz'' serian seriales correctos.


\subsection{Obteniendo ayuda}
Si tenemos alguna duda lo primero que podemos hacer es ejecutar \emph{man gdb}, si queremos un resumen de los comandos podemos ejecutar \emph{gdb --help} o \emph{gdb -h}. Si tenemos dudas más genéricas una vez dentro de GDB el comando \emph{help} puede servir. Sino podemos mirar la documentación de GDB con el comando \emph{info gdb}. \\
Si no encontramos nada podemos ir a su web oficial \url{https://www.gnu.org/software/gdb/} donde hay información sobre el canal IRC y las listas de correo que también pueden ser de utilidad.\\
GDB tiene un sistema de auto-completado de comandos con la tecla \textbf{TAB}, suele venir muy bien para los vagos que no quieren escribir mucho o no recuerdan como se escribía un comando.


\subsection{Resumen}
En esta sección vimos unos cuantos ejemplos donde usamos GDB para diferentes tareas. Desde luego aún no sabemos nada sobre GDB pero ya le perdimos el miedo. En las siguientes secciones profundizaremos sobre como usar los comandos mencionados anteriormente y explicaremos algunas nuevas funcionalidades básicas de GDB. 
\section{Iniciando GDB}
Anteriormente ya vimos como cargar un ejecutable con GDB pero aún no sabemos como podemos debugear un proceso en ejecución, ni que ficheros carga al iniciar, etc. En esta sección veremos otras opciones para depurar un programa.

Los códigos usados estarán en \textbf{ejemplos/3}.

\subsection{Ejecutable}
Como vimos para cargar un ejecutable en GDB se lo podemos especificar como parámetro. También se puede usar el comando \emph{file} una vez iniciado GDB. 

En estos dos casos el ejecutable es cargado y los símbolos son leídos. Es equivalente a usar el flag \textbf{-se} que lee los símbolos del archivo y lo carga como ejecutable.\\
El flag \textbf{-s} o \textbf{-symbols} solo lee los símbolos, y el flag \textbf{-e} o \textbf{-exec} especifica que archivo se ejecutara pero no lee los símbolos.


\subsection{Proceso en ejecución}
Otra opción es depurar un proceso en ejecución especificando su PID (identificador de proceso). \\
Para este ejemplo usaremos el programa \textbf{ejemplos/3/attach.c}. Dicho código imprime su PID, y luego espera algún input, finalmente dice ''Bye'' y sale.

Así que compilamos y ejecutamos el binario, en este caso el PID que nos dice es ''5146''.
Dejamos el programa esperando el input y en otra consola abrimos GDB donde el primer parámetro es la ruta al ejecutable y el segundo el PID. Por defecto GDB si se le especifica un valor numérico primero comprobara si es un proceso, si no es un proceso probara si es un archivo ''core''.
\begin{verbatim}
$ gdb -q attach 5146
Reading symbols from attach...(no debugging symbols found)...done.
Attaching to program: /tmp/codes/2/attach, process 5146
Reading symbols from /lib/x86_64-linux-gnu/libc.so.6...(no debugging symbols found)...done.
Reading symbols from /lib64/ld-linux-x86-64.so.2...(no debugging symbols found)...done.
0x00007f26d5f5a8e0 in read () from /lib/x86_64-linux-gnu/libc.so.6
(gdb) 
\end{verbatim}
Como vemos primero lee el ejecutable y luego se une al PID especificado.\\
Ahora ya podemos trabajar como de costumbre. \\

Obviamente si intentamos debugear un proceso al cual no tenemos permisos nos dará un error. Intentara leer el ejecutable pero no tendrá privilegios necesarios, luego intentara unirse al PID y no podrá, y finalmente mirara si ese valor numérico era un fichero ''core'' en vez de un PID pero tampoco lo encontrara. 
\begin{verbatim}
gdb -q ./attach 8619
./attach: Permiso denegado.
Attaching to process 8619
ptrace: Operación no permitida.
/tmp/ejemplos/3/8619: No existe el fichero o el directorio.
(gdb)
\end{verbatim}

Al igual que en el caso anterior, también se puede attachear al programa usando el comando \emph{attach} (previamente abriendo el ejecutable).\\

También se existen las opciones \textbf{-pid} o \textbf{-p} que a parte de unirse (attach) al proceso que se esta ejecutando busca el ejecutable y lo lee.


\subsection{Símbolos de depuración}
Los símbolos de depuración pueden estar en un mismo ejecutable o venir a parte. Estos símbolos normalmente son stripeados (eliminados) del binario para reducir su tamaño y/o hacer más difícil su análisis. \\

Para la prueba usaremos el programa \textbf{ejemplos/3/attach.c} y las herramientas \emph{strip} y \emph{objcopy}, la primera para eliminar estos datos y la segunda para copiarlos a otro fichero..\\

Primero deberemos compilar con el flag \textbf{-ggdb} como de costumbre. Luego usaremos la herramienta \emph{objcopy} para extraer algunos símbolo a otro archivo. Ejecutamos \emph{objcopy --only-keep-debug attach attach.dbg} y nos creara el archivo attach.dbg con los símbolos. Luego eliminamos los símbolos del binario original, para esto podríamos usar la misma herramienta (objcopy) o usar \emph{strip}. Una vez ejecutemos el comando \emph{strip --strip-debug attach} ya tendremos los dos archivos.\\
Ahora ya podemos abrir el fichero ejecutable por una parte y cargar los símbolos por otra. En GDB hay los flags \textbf{-s} y \textbf{-symbols} y el comando \emph{symbol-file} para cargar los símbolos.\\
Si intentamos abrir GDB son el ejecutable nos dirá que no encontró los símbolos porque fueron eliminados y copiados en el archivo attach.gdb.\\
Se podría pensar que un comando como \emph{gdb -q attach -symbols attach.dbg} debería leer los símbolos, pero si nos fijamos no los lee.
\begin{verbatim}
gdb -q attach -symbols attach.dbg
Reading symbols from attach...(no debugging symbols found)...done.
(gdb)
\end{verbatim}
Eso es porque implícitamente intenta leer los símbolos de depuración del ejecutable, pero no los encuentra porque los borramos. Para que no los intente leer del ejecutable podríamos especificar cual es el archivo ejecutable y cual el de los símbolos. Por ejemplo ejecutando \emph{gdb -q -exec attach -symbols attach.dbg}. Como dijimos anteriormente el flag \textbf{-exec} servía para especificar el ejecutable.
\begin{verbatim}
gdb -q -exec attach -symbols attach.dbg
Reading symbols from attach.dbg...done.
(gdb)
\end{verbatim}
Como vemos ya los leyó correctamente.


\subsection{Archivo ''core''}
A veces los errores mortales generaran ''core files''. Esos ficheros guardan una imagen del proceso cuando murió y pueden ser usados por GDB u otras herramientas para analizar dicho ejecutable e intentar averiguar que causo su muerte. \\
Muchas veces este archivo core es generado automáticamente después de un fallo grave. Si no es generado automáticamente, se puede generar uno con GDB usando el comando \emph{generate-core-file} o \emph{gcore} donde el primer argumento es el nombre del archivo.\\

Para este ejemplo usaremos el code \textbf{ejemplos/3/dead.cpp}. Es un simple programa en C++ que peta porque se intenta desreferenciar un puntero nulo.\\
Luego usando el flag \textbf{-c} o \textbf{-core} o usando el comando \emph{core-file} leemos el archivo core que en este caso se llama ''dead.core''. 
\begin{verbatim}
$ gdb -q dead -core dead.core
Reading symbols from attach...(no debugging symbols found)...done.

[New LWP 8888]
Core was generated by `/tmp/ejemplos/3/dead'.
Program terminated with signal SIGSEGV, Segmentation fault.
#0  0x00000000004009eb in ?? ()
(gdb)
\end{verbatim}
Como vemos nos indica que se produjo un SIGSEV y en que dirección se produjo.
Luego podemos desensamblar el punto donde peto y mirar que estaba haciendo.
\begin{verbatim}
(gdb) disas main
Dump of assembler code for function main:
...
0x00000000004009d3 <+61>:	mov    rax,QWORD PTR [rbp-0x8]
0x00000000004009d7 <+65>:	mov    rdi,rax
0x00000000004009da <+68>:	call   0x400800 <_ZdlPv@plt>
0x00000000004009df <+73>:	mov    QWORD PTR [rbp-0x8],0x0
0x00000000004009e7 <+81>:	mov    rax,QWORD PTR [rbp-0x8]
=> 0x00000000004009eb <+85>:	movzx  eax,BYTE PTR [rax]
...
(gdb) 
\end{verbatim}
Como vemos primero copió \textbf{0x0} a \textbf{rbp-0x8} y luego lo copió a \textbf{rax}. RAX ahora vale 0, así que al intentar obtener el valor apuntado por RAX peto.


\subsection{Archivos de GDB}
Al iniciar, GDB lee una serie de archivos de configuración que pueden servir, por ejemplo, para que por defecto se use la sintaxis de intel y no la de at\&t
El archivo \texttt{\$HOME/.gdbinit} se procesa antes que las opciones del comando. Obviamente este archivo de configuración es global para el usuario. \\
Si el archivo \texttt{gdibit} esta en la carpeta actual, este fichero se leerá después de procesar las opciones del comando, con la excepción de \textbf{-x} y \textbf{-ex} que se procesan después. Depende del caso puede existir otro archivo que se carga antes que los comandos, que los otros ficheros y que es global para todo el sistema. \\

Si preferimos la sintaxis intel podemos escribir el comando que dijimos anteriormente (\emph{set disassembly-flavor intel}) en el archivo \texttt{\$HOME/.gdbinit}. Más tarde veremos como podemos jugar más con estos ficheros.

En caso de que algún momento no queramos que estos archivos sean leídos podemos usar el flag \textbf{-nh} para omitir el archivo \texttt{\$HOME/.gdbinit} o los flags \textbf{-n} o \textbf{-nx} que omiten todos los archivos \texttt{.gdbinit}



%\section{Breakpoints}
Un breakpoint es una orden que damos al depurador para que nos pause la ejecución en un lugar concreto. GDB soporta 3 tipos de breakpoints, los \emph{breakpoints}, \emph{watchpoints}, y \emph{catchpoints}.\\
Los \emph{breakpoints} detienen el programa cuando llega a un cierto punto, estos pueden tener condiciones. Los \emph{watchpoints} (o \emph{data breakpoints}) detienen el programa cuando se accede a una cierta zona de datos, por ejemplo cuando una variable es leída y/o escrita. Los \emph{catchpoints} detienen el programa cuando se da un evento concreto, por ejemplo al cargar una librería.\\

Para breakpoint que pongamos GDB le pondrá un número. Este identificador numérico servirá para deshabilitarlos, borrarlos o hacer operaciones con rangos.\\
Para listar todos los breakpoints se puede usar el comando \emph{info breakpoints}, para listar solo los watchpoints el comando a usar sería \emph{info watchpoints}.

\subsection{Breakpoints} %0xcc
El código que se usara esta en \textbf{ejemplos/4/breakpoints.c}.

Hay 2 formas de poner estos breakpoints, por software o por hardware. Los ''software breakpoints'' son más flexibles y tienen menos limitaciones que los ''hardware breakpints'' pero son más fáciles de detectar. Las técnicas antid-debug las veremos más adelante.\\
\emph{NOTA: Si solo decimos ''breakpoints'' nos referimos a software breakpoints}\\

La forma más rápida de colocar un breakpoint es usando el comando \emph{break} tal y como vimos anteriormente. El argumento puede ser el nombre de una función, un número de línea o una dirección. Si no se especifica ningún parámetro pondrá el breakpoint en la siguiente instrucción a ser ejecutada. También se puede especificar una condición, por ejemplo: si quisiéramos poner un breakpoint al llegar a la interacción número 100, y esta compilamos con debugging symbols podríamos hacer algo tan simple como \emph{break *main+21 if i==100}. Si luego imprimimos la información de los breakpoints veremos 
\begin{verbatim}
(gdb) i b
Num     Type           Disp Enb Address            What
1       breakpoint     keep y   0x00000000004004cb in main at breakpoints.c:5
       stop only if i == 100
(gdb) 
\end{verbatim}
Si no tuviéramos esa información de depuración igualmente podemos usar breakpoints condicionales. Aunque quizás no sea tan intuitivo.
\begin{verbatim}
(gdb) disas main
Dump of assembler code for function main:
0x00000000004004b6 <+0>:	push   rbp
0x00000000004004b7 <+1>:	mov    rbp,rsp
0x00000000004004ba <+4>:	mov    DWORD PTR [rbp-0x4],0x0
0x00000000004004c1 <+11>:	jmp    0x4004c7 <main+17>
0x00000000004004c3 <+13>:	add    DWORD PTR [rbp-0x4],0x1
0x00000000004004c7 <+17>:	cmp    DWORD PTR [rbp-0x4],0x63
0x00000000004004cb <+21>:	jle    0x4004c3 <main+13>
0x00000000004004cd <+23>:	mov    eax,0x0
0x00000000004004d2 <+28>:	pop    rbp
0x00000000004004d3 <+29>:	ret    
End of assembler dump.
(gdb) b *main+21 if *(int*)($rbp-0x4) == 100
Breakpoint 1 at 0x4004cb
(gdb) 
\end{verbatim}
Como vemos la variable \textbf{i} en la máquina equivale a la dirección (\$rbp-0x4). Puesto que esa dirección apuntara a un int, lo podemos ''castear''. Como no queremos comparar con la dirección sino con el valor al que apunta usamos el operador \textbf{*}. Así que la comparación finalmente queda \emph{*(int*)(\$rbp-0x4) == 100}.\\
También podemos añadir la condición después de colocar el breakpoint con el comando \emph{condition}. Este comando tiene 2 argumentos, el primero es el número de breakpoint, y el segundo la condición. 
\begin{verbatim}
(gdb) break *main+21
Breakpoint 1 at 0x4004cb
(gdb) condition 1 *(int*)($rbp-0x4) == 100
(gdb) i b
Num     Type           Disp Enb Address            What
1       breakpoint     keep y   0x00000000004004cb <main+21>
stop only if *(int*)($rbp-0x4) == 100
(gdb)
\end{verbatim}

Para colocar breakpoints temporales podemos usar el comando \emph{tbreak}. Este comando funciona igual que el comando break solo que una vez usado se borra automáticamente. \\

El comando \emph{hbreak} permite colocar breakpoints por hardware. Al depender del hardware el número de breakpoints es limitado, y en algunos casos puede que el hardware no lo permite. El uso de este comando es igual que en los dos anteriores casos.\\
Si el comando \emph{tbreak} eran breakpoints temporales y el comando \emph{hbreak} eran breakpoints por hardware, el comando \emph{thbreak} son breakpoints temporales por hardware.\\

El comando \emph{rbreak} es algo distinto. Este comando permite poner varios breakpoints incondicionales que cumplan un regexp (como el que usa ''grep'').  Por ejemplo para poner un breakpoint en todas las funciones que empiecen por ''str'' se puede usar el comando \emph{rbreak \textasciicircum str}. Si queremos limitar la búsqueda a un archivo podemos especificar el nombre del archivo seguido de dos puntos antes del ''regexp'', por ejemplo \emph{rbreak archivo.c:\textasciicircum str}


\subsection{Watchpoints} 
El código que se usara esta en \textbf{ejemplos/4/watchpoints.c}.

Hay 3 formas de colocar estos watchpoints. Se puede poner un watchpoint al leer una variable, al escribir en ella o en cualquiera los dos casos. GDB intentara colocar hardware breakpoints en vez de software breakpoints para los watchpoints  si es posible.

Para poner un watchpoint cuando una variable sea escrita (modificada) usaremos la orden \emph{watch} seguida de una expresión. Este comando adicionalmente acepta 2 parámetros, el número de hilo y una máscara. El threadnum (número de hilo) sirve para especificar que solo vigile un thread en especifico, este caso solo funciona con los hardware breakpoints. La máscara sirve principalmente para poder poner varios watchpoints de forma simultanea, aunque en este texto no le daremos demasiada importancia ya que los PCs de ir por casa (arquitecturas x86) no lo soportan. \\
Por ejemplo si tenemos la variable \textbf{i} y a \textbf{x}, que en el inicio valen 0 y ponemos un \emph{watch i*x}
\begin{verbatim}
(gdb) watch i*x
Hardware watchpoint 2: i*x
(gdb) c
Continuing.
123
Hardware watchpoint 2: i*x

Old value = 0
New value = 12300
0x00007ffff7a8f5ea in _IO_vfscanf () from /lib/x86_64-linux-gnu/libc.so.6
(gdb) 
\end{verbatim}
Podemos ver que cuando el programa entra en el scanf y nos pide el nuevo valor de \textbf{x}, si le introducimos algo distinto que cero saltara el watchpoint y nos avisara del valor antiguo y del nuevo. Si hubiéramos ingresado un 0, i*0 sigue siendo 0 y no habría saltado. \\
También podemos poner watchpoints en valores booleanos, por ejemplo
\begin{verbatim}
(gdb) watch i*x==500
Hardware watchpoint 2: i*x==500
(gdb) c
Continuing.
5
Hardware watchpoint 2: i*x==500

Old value = 0
New value = 1
0x00007ffff7a8f5ea in _IO_vfscanf () from /lib/x86_64-linux-gnu/libc.so.6
(gdb)
\end{verbatim}
Como vemos el valor antiguo era 0 (falso) y el nuevo 1 (cierto) ya que al acabar el bucle la variable \textbf{i} vale 100, y \textbf{x} le cambiamos el valor por 5.\\

Para poner un watchpoint al leer un valor usaremos el comando \emph{rwatch} y para ponerlo en lectura o escritura usaremos \emph{awatch}. El funcionamiento es idéntico al caso anterior. \\
 
El programa de ejemplo que vamos a usar se puede encontrar en \textbf{ejemplos/4/rwatch.cpp} y lo podemos compilar con \textbf{g++ rwatch.cpp -o rwatch -std=c++11 -pthread}. Este programa tiene un error por el que la mayoría de veces se quedara en un bucle infinito.

Una vez tengamos cargado el ejecutable en gDB podemos poner un break en la función main y ir avanzando en la ejecución para ver como se crean los dos threads.
\begin{verbatim}
Reading symbols from rwatch...(no debugging symbols found)...done.
(gdb) tbreak main
Temporary breakpoint 1 at 0x400ef7
(gdb) r
Starting program: /home/david/Soft/Proyectos/fennix-GDB/src/ejemplos/4/rwatch 
[Thread debugging using libthread_db enabled]
Using host libthread_db library "/lib64/libthread_db.so.1".

Temporary breakpoint 1, 0x0000000000400ef7 in main ()
(gdb) nexti
...
(gdb) 
[New Thread 0x7ffff7008700 (LWP 10771)]
....
(gdb) 
[New Thread 0x7ffff6807700 (LWP 10815)]
0x0000000000400f1e in main ()
(gdb)
\end{verbatim}
Aquí vemos dos valores LWP, estos números son identificadores de la tarea. Si ejecutamos el comando \textbf{thread apply all bt} podremos ver el backtrace de todos los threads para mirar que función estaban ejecutando, en este caso de los 3.

\begin{verbatim}
(gdb) thread apply all bt

Thread 3 (Thread 0x7ffff6807700 (LWP 10815)):
#0  0x00007ffff73b42bd in nanosleep () from /lib64/libpthread.so.0
#1  0x000000000040157c in void std::this_thread::sleep_for<long, std::ratio<1l, 1000l> >(std::chrono::duration<long, std::ratio<1l, 1000l> > const&) ()
#2  0x0000000000400ee1 in f2() ()
#3  0x0000000000402711 in void std::_Bind_simple<void (*())()>::_M_invoke<>(std::_Index_tuple<>) ()
#4  0x0000000000402659 in std::_Bind_simple<void (*())()>::operator()() ()
#5  0x00000000004025d6 in std::thread::_Impl<std::_Bind_simple<void (*())()> >::_M_run() ()
#6  0x00007ffff7b86980 in execute_native_thread_routine () from /usr/lib/gcc/x86_64-pc-linux-gnu/4.9.3/libstdc++.so.6
#7  0x00007ffff73ab644 in start_thread () from /lib64/libpthread.so.0
#8  0x00007ffff70f0edd in clone () from /lib64/libc.so.6

Thread 2 (Thread 0x7ffff7008700 (LWP 10771)):
#0  0x00007ffff73b42bd in nanosleep () from /lib64/libpthread.so.0
#1  0x000000000040157c in void std::this_thread::sleep_for<long, std::ratio<1l, 1000l> >(std::chrono::duration<long, std::ratio<1l, 1000l> > const&) ()
#2  0x0000000000400e9f in f1() ()
#3  0x0000000000402711 in void std::_Bind_simple<void (*())()>::_M_invoke<>(std::_Index_tuple<>) ()
#4  0x0000000000402659 in std::_Bind_simple<void (*())()>::operator()() ()
#5  0x00000000004025d6 in std::thread::_Impl<std::_Bind_simple<void (*())()> >::_M_run() ()
#6  0x00007ffff7b86980 in execute_native_thread_routine () from /usr/lib/gcc/x86_64-pc-linux-gnu/4.9.3/libstdc++.so.6
#7  0x00007ffff73ab644 in start_thread () from /lib64/libpthread.so.0
#8  0x00007ffff70f0edd in clone () from /lib64/libc.so.6

Thread 1 (Thread 0x7ffff7fc7740 (LWP 10688)):
#0  0x0000000000400f1e in main ()
(gdb)
\end{verbatim}
Luego podríamos poner un breakpoint en la variable \emph{global} en el thread 3 (f2) y ver que se esta leyendo.
\begin{verbatim}
(gdb) rwatch global thread 3
Hardware read watchpoint 2: global
(gdb) c
Continuing.
[Thread 0x7ffff7008700 (LWP 10771) exited]
[Switching to Thread 0x7ffff6807700 (LWP 10815)]

Thread 3 "rwatch" hit Hardware read watchpoint 2: global

Value = 1
0x0000000000400ee1 in f2() ()
(gdb)
\end{verbatim}
Como vemos f2 leyó de la variable \emph{global} el valor 1 y esperaba un 2 para finalizar el thread.\\

Para ver los breakpoints colocados podemos usar la orden \emph{info breakpoints} o si solo queremos ver watchpoints \emph{info watchpoints}


\subsection{Catchpoints}
Los catchpoints nos permiten pausar la ejecución al producirse algún evento como una excepción de C++, una llamada a sistema, al cargar una librería dinámica, etc. Por ejemplo con el mismo ejecutable que en el punto anterior podríamos poner un catch al ejecutar la función \emph{clone} (syscall 56).
\begin{verbatim}
(gdb) catch syscall 56
Catchpoint 1 (syscall 56)
(gdb) r
Starting program: /home/david/Soft/Proyectos/fennix-GDB/src/ejemplos/4/rwatch 
[Thread debugging using libthread_db enabled]
Using host libthread_db library "/lib64/libthread_db.so.1".

Catchpoint 1 (call to syscall 56), 0x00007ffff70f0ea1 in clone () from /lib64/libc.so.6
(gdb) 
\end{verbatim}


\subsection{Eliminando y deshabilitando breakpoints}
Para eliminar breakpoints de cualquiera de estos 3 tipos podemos usar la orden \emph{delete}. Si no se le especifica ningún argumento borrara todos los breakpoints. Si se le especifica un número \textbf{N}, borrara el breakpoint \textbf{N}. Si se le especifica un rango \textbf{N-M} eliminara todos los breakpoints desde el \textbf{N} hasta el \textbf{M} (\textbf{N} y \textbf{M} incluidos): Por ejemplo, la orden \texttt{delete 5-7} borrara los breakpoints 5, 6 y 7.

Para desactivar los breakpoints en vez de borrarlos existe el comando \texttt{enable} y \texttt{disable}. Como parámetros se les puede especificar un rango o el número del breakpoint. La orden \texttt{enable} también permite activar un breakpoint para que automáticamente tras saltar sea eliminado, deshabilitado, o bien ponerle un contador a un breakpoint para que al cabo de X acciones (aka ''hits'') se desactive. Por ejemplo:
\begin{verbatim}
(gdb) disas main
Dump of assembler code for function main:
0x0000000000400526 <+0>:	push   rbp
0x0000000000400527 <+1>:	mov    rbp,rsp
0x000000000040052a <+4>:	sub    rsp,0x10
0x000000000040052e <+8>:	mov    DWORD PTR [rbp-0x8],0x0
0x0000000000400535 <+15>:	mov    DWORD PTR [rbp-0x4],0x0
0x000000000040053c <+22>:	jmp    0x400542 <main+28>
0x000000000040053e <+24>:	add    DWORD PTR [rbp-0x4],0x1
0x0000000000400542 <+28>:	cmp    DWORD PTR [rbp-0x4],0x63
0x0000000000400546 <+32>:	jle    0x40053e <main+24>
...
0x000000000040055e <+56>:	mov    eax,0x0
0x0000000000400563 <+61>:	leave  
0x0000000000400564 <+62>:	ret    
End of assembler dump.
(gdb) i b
Num     Type           Disp Enb Address            What
1       breakpoint     keep y   0x0000000000400546 in main at watchpoints.c:6
2       breakpoint     keep y   0x0000000000400542 in main at watchpoints.c:6
3       breakpoint     keep y   0x000000000040053e in main at watchpoints.c:6
(gdb)
\end{verbatim}
Podemos poner que el breakpoint 1 se desactive al cabo de 5 ''hits'', el 2 se deshabilite en el primer ''hit'' y el 3 se borre al primer ''hit''.
\begin{verbatim}
(gdb) enable once 2
(gdb) enable delete 3
(gdb) enable count 5 1
(gdb) i b
Num     Type           Disp Enb Address            What
1       breakpoint     dis  y   0x0000000000400546 in main at watchpoints.c:6
disable after next 5 hits
2       breakpoint     dis  y   0x0000000000400542 in main at watchpoints.c:6
3       breakpoint     del  y   0x000000000040053e in main at watchpoints.c:6
(gdb) 
\end{verbatim}
Si ahora damos run y volvemos a mirar los breakpoints veremos como el 3 ya no existe y el 2 se deshabilito.
\begin{verbatim}
(gdb) i b
Num     Type           Disp Enb Address            What
1       breakpoint     dis  y   0x0000000000400546 in main at watchpoints.c:6
breakpoint already hit 1 time
disable after next 4 hits
2       breakpoint     dis  n   0x0000000000400542 in main at watchpoints.c:6
breakpoint already hit 1 time
(gdb)
\end{verbatim}

\begin{colabs}
stakewinner00 : stakewinner00@mykolab.com
\end{colabs}

\end{document}
